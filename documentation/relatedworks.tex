\section{Related Work}

We present relevant apects of selected work in the anonymity and
distributed filesharing domains from which we have taken inspiration.

\subsubsection*{Why Johnny can't stay anonymous}

Johnny can't encrypt if the secure applications available are non-intuitive and difficult for
less technical users to properly employ. Whitten et al discovered by way of lab trials that PGP's
user interface was difficult to use and understand even for people with some technical background,
at time of publication.

While user experience overall has made much progress and is one of the foremost design
concerns in building `everyday' applications, usability is still less frequently considered when applications,
tools, and frameworks are developed with user privacy and security as immediate concerns. If Johnny has information he needs
to get to Alice without people with a lot of computing power reading over
his shoulder, he should not need to become an expert in systems-level security to do so. Anyone ought to be able to properly configure
privacy-providing services and encrypt, sign, send, and receive data with no possibility of interference
by any third parties (including the service provider) and little fuss.

Suppose Johnny is a journalist, activist, or anyone with little technical expertise who might be working
somewhere with business practices on the shady side that are actively hurting people.
If services such as anonymous filesharing are difficult to understand and use, Johnny might not be able to get the
word out. He might even be laid off, detained, or arrested if visible use of such
services is against local law or company policy. We
propose creating a simple client-facing application and making
its source code available for public, version-controlled
contribution on Github.

\subsubsection*{Tor and the onion protocol}

Tor is a widely used, well-documented project claiming to provide more security than its users would have openly
conducting business. The Tor FAQ states that peer-to-peer filesharing is not desired (or to be a focus of operation)
within the Tor network. Data such as IP addresses sent over UDP-based P2P protocols are not anonymized by Tor or
SOCKS proxies in general. Though some thought in recent years has been given to including UDP support in Tor both
to improve latency issues and to allow for the inclusion of UDP-based data transport methods, to
our knowledge the Tor developers have not started serious work on or released such an improvement to the network.

Further, due (we hypothesize) to Tor's notoriety in mainstream media and additionally the general public's lack of
understanding of Tor's inner workings, the number of people who contribute back by running nodes
or improving the codebase is much smaller than the number of people merely using the service to protect themselves.
We find this situation analogous to most P2P networks' leechers-versus-seeders situation, in which those who download
and then do not actively seed typically outnumber those that download and then seed files to others.
While the value of Tor does increase with the number of users to provide `cover' to one's
own traffic, the network itself still remains small in proportion to the number of users.
The number of those who run exit nodes (a crucial part of the Tor infrastructure) is smallest of all subsets of Tor
volunteers, likely due to a history of abuse by malicious users and backlash from ISPs.
Systems such as TorWard are being developed to assess and reduce the
risk of ill-meaning traffic to exit node operators and other Tor
volunteers, but work in this space is relatively nascent.

Tor uses an incremental path-building design to connect Alice to Bob:
each successive hop in a \textit{circuit}, which is a path between a user's entry
point and her destination, negotiates its own session keys to provide each acting node in a circuit a sort of plausible
deniability. In our own design, transport of a given chunk of data from
the user to the chosen destination is modelled closely
after these circuits. With Tor, both endpoints must be synchronized for connection to occur, where in our design, data transport
is effectively \textit{asynchronous}: data will only be received when a request is sent for it.

In addition to letting users browse relatively securely,
Tor offers its users the ability to set up \textit{hidden services}, or alternatives to sites on the open web offering anything
servable over a SOCKS proxy. While the idea is that operating such a service would be anonymous in that the operator's
IP address, name, physical location, and other personal details should not be revealed, it has been shown that this isn't
the most reliable or safe practice. Hidden services, by way of a \textit{hidden service descriptor},
which consists of the service's public key and a list of its
acknowledged introduction points (relays that act essentially like human translators in relaying traffic between site
accessors and the site itself) can advertise themselves in a globally available directory.

Furthermore it has been shown using the Cisco traffic analysis tool Netflow that it is possible to uncloak around 81\% of
Tor users, particularly in cases where the users did not properly configure their setups. While we strongly advocate for
an improved Tor, we believe our system combines its better points with the ease of peer-to-peer transactions.

\subsubsection*{I2p}

Somewhat like Tor in purpose, I2p is an anonymous network based around the \textit{I2p router}. I2p is designed to work
atop the bespoke UDP and TCP-like protocols SSU and NTCP. These protocols run atop the peer-to-peer network, which itself
sits on top of IP. All peer and service metadata is stored in a distributed hash table called
\textit{netDB}. Network database records are stored in a Kademlia-like DHT \textit{floodfill}.
Since netDB is run alongside regular I2p nodes, it may be considered less a trusted
group of authorities than functionally interwoven in the network.

All connections inside I2p's inclusive \textit{darknet} are end-to-end encrypted; the router multiplexes theses connections over
pairs of single-direction \textit{I2p tunnels}, which are (when considered as a pair) analogous to Tor's circuits. These tunnels
are constructed with the onion protocol. All participants are meant to remain anonymous, though in practice this does not always
appear to be the case. I2p is a much smaller project than Tor; this means fewer users providing white noise to mask one's traffic,
fewer developers, and fewer volunteers running nodes and services.

Additionally, at time of writing the port of the secure distributed Tahoe filesystem allowing decentralized storage similar
to Freenet on top of I2p is not maintained, and is not up to date with the current version of the filesystem. The largest difference
between Tahoe and Freenet, according to Tahoe's creators, is that file distribution in Freenet is totally random and that secure
usage of Tahoe depends on having a large network of friends and family one can shoehorn into also using the system, since Tahoe assumes
discovery of storage happens out-of-band. Further, Tahoe makes no attempt to obscure the source or destination of
requests. While it is exceedingly difficult to guarantee a
high probability of remaining anonymous against sophisticated attackers, that doesn't mean trying to (and explicitly stating that
use of the service is at one's own risk) isn't worthwhile.

\subsubsection*{Freenet}

Freenet is a location-independent peer-to-peer distributed filesystem of nodes that collaborate to store and retrieve encrypted files,
which are named by 160-bit SHA1 hashes that do not depend on the files' locations. Each user maintains their own \textit{local} datastore
contribution that stores a randomized assortment of other users' data as a portion of their hard drive (which might otherwise go
under-utilized). Freenet maintains a \textit{hops-to-live} value analogoous to
the ICMP time-to-live on active requests for files, which is decremented at each node the request reaches, to prevent infinite cycling
through the system.

We improve on the somewhat risky-seeming practice of safeguarding others' data on one's own machine by moving our
users' data storage contributions offsite to virtual private storage instances.
Additionally we repurpose the \textit{hops-to-live} idea to additionally limit the number of
downloads that may take place before the space in our distributed filesystem dedicated to a file's chunks and identifier will be
repurposed to hold other information (Freenet uses an LRU caching policy to determine whhat gets deleted when from an individual node
when there is not enough space).

\subsubsection*{CYRUS} % talk more about sangtae's paper

CYRUS is a secure data-sharing system to be used amongst one's own devices. One of the things most interesting about CYRUS is it relies
upon the concept of \textit{secret sharing}: that is, a piece of data is broken into chunks and can be reconstructed from any of those
chunks. We believe this is both highly useful for fault tolerance and incredibly insecure; for now we have chosen to improve on this
method by encrypting our data before it is chunked and sent into our distributed hash table, but future refining of this idea is highly
possible.

Another notable feature of CYRUS is its simple, intiuitive, and clean user interface.
Currently it can only be used on iOS, which we believe is a bit
of a drawback, but porting to other operating systems and so on is a relatively easy task. We chose (instead of creating a native
application) to write our clientside interface entirely in Javascript, specifically node.js, so that it can simply be opened locally
in any browser. While others may disagree with our choice to not host this application ourselves, we believe it is more secure for users
to simply start the application, point their browser at localhost, and therefore have their data encrypted and chunked before
it leaves their local machine.

\subsubsection*{BitTorrent} % and kademlia tables

BitTorrent's internal distributed hash table extension, which is based on the Kademlia model and allows for the tracking of peers without
a standard tracker, initially inspired much of this work, though the BitTorrent routing protocol (and the DHT extension) are far more
complicated than would suit our needs. Further, trackerless torrent groups often require the disclosure of some identifying
information to join, are not widely publicised, and are typically used for distribution of material copyrighted by others.

We do not want to employ the idea of ``peers" except between nodes in our DHT: our hash table
is essentially a multiway peer-to-peer network, tracked by a globally (publicly) available ``peer list" that is updated by each new
node when it joins the content store. We utilize the idea of ``trackerless torrents" as well to leave less of a paper trail for data
transfers between nodes, which are designed to firstly ensure no single node knows exactly what it is storing and secondly to keep
the sum total of the content in the data store balanced across all currently available nodes.

% talk about Kademlia and how it works

% talk more about how BT uses it?

\subsubsection*{Bamboo} %super sweet dht which we are going to use

Bamboo (OpenDHT's distributed hash table) relies on a routing algorithm called Chord-FRT. While we originally planned to implement our
own distributed hash table and routing algorithm, Chord-FRT did nearly everything we planned to include in our version. Commands include
GET, PUT, and RM: GET requests a chunk of data to the table, PUT adds data, and an RM request will (eventually) remove a file once the
node to which the RM was issued comes in contact with a piece of that file.

We chose to organize our chunks of information in each node of the DHT via a key,value data store. We improve upon the security of adding
items to and removing items from Bamboo by implementing onion routing from the client to an item's eventual resting place: each ``hop",
from the client to the edge node it has made initial contact with, and from that edge node through the DHT, negotiates its own session keys
and removes a layer of the ``onion" from the data chunk as the chunk is passed through. We will speak more in detail to this later.

