\section{Introduction}
\label{sec:intro}

In researching and constructing this system, we had three main
objectives in mind.

\subsubsection*{Security}

While the Internet is a multinational, borderless entity, its
cultural heritage is based in the expectation that \textit{all}
users may speak freely and exchange information and ideas with
anyone, at any time. Tor, Freenet, and I2P all claim to
allow their users to transfer and/or receive information without
exposure of real-world credentials such as legal names and
IP addresses. These services are utilized across demographics
and national boundaries. This speaks to a strong desire for
uncensored communication and information sharing.

The United Nations' Universal Declaration of Human Rights details
certain freedoms to which all individuals are entitled. Its
twelfth and nineteenth articles in particular, similarly to the
First Amendment to the United States Constitution and Section 2b
of the Canadian Charter of Rights and Freedoms, paraphrased,
state that all humans should have the
right to freedom of expression and
opinion and should not be subject to arbitrary interference with
their assembly and privacy.

In the interest of a more open Internet where information may be
shared freely regardless of national and political differences,
security and user privacy must be primary considerations for
developers whose applications deal with users' personal data.
When security is not a bolt-on afterthought but is initially
important, one's work becomes more difficult to compromise.

\subsubsection*{Usability}

We are also concerned with the ability of our users to easily
understand and employ our service without accidentally revealing
personally identifiable information or other data they do not
want shared. Our system should offer the fewest (unpleasant)
surprises possible to the user both in terms of how the interface
works and that their data is actually secure.

While Tor and similar services were designed with some measure
of usability in mind, we argue that it is quite possible to do
better, particularly in light of recent breaches of cloud
filesharing services as well as crackdowns on Tor hidden service
operators.

The general public has been trained to expect the convenience of
being able to launch any sort of application with little fuss.
Popular virtual private storage / storage as a
service providers do little to contradict (or even prove correct
by implementing and continuously improving more secure systems)
the belief that properly securing data is trivial,
or that one's data will be secure at
all once entrusted to the service provider.

With Tor particularly, it is easy to misconfigure some aspect
of one's personal browsing or hidden service setup without
at least some knowledge of the inner workings of Internet
security and privacy technologies.

Further, Snowden argues that some possible adversaries are
capable of a trillion hashes per second. It therefore is logical
to make our system as simple, transparent, and \textit{ironclad}
as we possibly can. We use as much layered, strong encryption as
we can manage without overcomplicating the user experience (or the
back end code) as well as a few lessons learned from the best
points of Tor, I2p, and Freenet to do our best to deliver to
users' expectations that the integrity of their data has not
been undermined.

\subsubsection*{Simplicity}

Each component should do just one thing. It should be simple
for future developers to explore and test new as well as existing
functionality, or even extend this work in new directions.
Each of our components should be designed and tested with limited
reliance on elements belonging to other components.
The simpler and more sturdy a system and its parts are, the easier
it is to reuse parts or extend them (and the more likely we
are to convince others to hack on and improve our ideas).

During the beginning of the implementation process, we realized
our design at the time used a lot of moving parts simply because
the technology was interesting, new, or partially solved a problem.
Byzantine faults are more likely in a more complicated system.
\textit{Failure} due to Byzantine faults is both common to many
sorts of distributed systems and potentially exploitable by
attackers. The less complicated and more streamline our system
becomes, the more easy it will be to implement redundancy and
guard against `benign' as well as malicious faults and failures.

%Probably also say something about DHT's here

