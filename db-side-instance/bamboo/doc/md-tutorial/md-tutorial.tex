\documentclass[twocolumn, twoside, a4paper]{article}

\usepackage{listings}
\usepackage{url}
\usepackage{verbatim}

\begin{document}

\title{\textsc{Bamboo} - A Tutorial}
\author{Marcel Dischinger\\
  \emph{(marcel.dischinger@tm.uka.de)}}
\date{April 2004}
\maketitle

\section*{Abstract}
\label{sec:Abstract}
%%===============================

The goal of this tutorial is to show you how to build a simple \textsc{Bamboo}
application that can send simple text messages from one node to
another. In addition it will give you a rough insight to the bamboo
communication and event system. We assume that the reader has some 
knowledge of P2P systems (e.g.~\cite{oceanstore})
before starting.
It cannot be our task to explain how a P2P system like
\textsc{Bamboo} works. Refer to the available documentation for
\textsc{Bamboo}~\cite{bamboo}.
and Pastry~\cite{pastry} to
learn more about how the \textsc{Bamboo} router works and how the P2P system
is organized. You also need to know how to write a programme in Java.

\section{Introduction}
\label{sec:Introduction}
%% ==============================

When I started working with \textsc{Bamboo} there was no tutorial.
There are some examples of modules in the source tree but they
are too complex and sometimes too inconsistent to learn from (beside a
lack of comments and documentation for the simple structures used at each
stage).

Bamboo is written in an event-driven, single-threaded programming
style. In first place it inherits its structure from SEDA, which
stands for the \emph{Staged, Event-Driven Architecture}. As the name
suggests, each part of a \textsc{Bamboo} application is a \emph{stage};
communication is done by passing events to each stage, while the whole
application remains single threaded and serialized.

This tutorial is meant to be a short introduction to
\textsc{Bamboo} without giving a deep insight of the underlying system. It might
prepare you to start writing your own stages using \textsc{Bamboo} and
to use the basic functions provided by the system to create your own
stage. 

This can only be a start. If you are going to write more complex
stages, you will end up reading the source code provided, but by that
time it will be easier because you know the basics. This Tutorial does
not deal with the \textsc{Bamboo} DHT API. 

\textsc{Bamboo} uses a lot of code from the Ocean\-Store project. You
can get the source code and Java-Doc from the tapestry
project~\cite{tapestry}, if you would like to take a look at it. You
do not need it for running \textsc{Bamboo}.

In this tutorial I also describe how to use the SEDA
event model which is implemented in \textsc{Bamboo}. A more \emph{modern} way
to do that would be to use the Async implementation (see ``Programmers
Guide'' on~\cite{bamboo}). 


\section{Structure of a stage}
\label{sec:structure}
%% ==============================

This section will give you a rough overview of the elements of a
stage and the usage of the config file. You will find more detailed
explanation in section~\ref{sec:walkthrough} where we will walk through
the code of the example appended and explain the how and
why more deeply, through that example.


%% ==============================
\subsection{First overview}
%% ==============================
\label{sec:first}

A new stage must implement the \texttt{seda.sandStorm.api.EventHandlerIF} interface in order to
act as a \textsc{Bamboo} stage and be part of its event handling. There is
also a class called \texttt{bamboo.util.StandardStage} you can inherit
from. We will do this for our new stage, called \emph{SimpleStage}
(see Appendix~\ref{sec:source} for the source code).

The stage can be seen as having three parts; each is refered to a
different point in time, during the lifetime of a \textsc{Bamboo} stage.

\begin{itemize}
\item \texttt{SimpleStage()} (the constructor)
\item \texttt{init(ConfigDataIF config)}
\item \texttt{handleEvent(QueueElementIF elem)}
\end{itemize}

The \emph{Constructor} should be used to initialize the data of the stage (as
you would do it in any Java programme). In \textsc{Bamboo} you need to subscribe
to events you want to hear from, so this is the place to fill the array
structures provided by \emph{StandardStage} with the events you need
and want.

The registering of the events is done in the \texttt{init()} function. This
function is called during startup, passing the configuration parameters
to us. The registration is be done by \emph{StandardStage}, so it is a
good idea to call \texttt{super.init()} here. We also want to fetch all
configuration parameters here, and it is a good place to register
payloads for our outgoing messages. 

\texttt{handleEvent()} is called from \textsc{Bamboo} every time a event we
registered to arrives and needs to be handled. This is the core of
our stage.

There is another function called \texttt{destroy()} in
\texttt{seda.sandStorm.api.EventHanderIF}. It was meant to be a
destructor but actually it is deprecated in \textsc{Bamboo} and is never
called.


\subsection{The configuration file}
\label{sec:config}
%% ==============================

One thing that \textsc{Bamboo} inherits from Ocean\-store is the Sand\-Storm
tool to parse the configuration file. The config file is XML-like and
very easy to understand as you will see.


\lstset{language=XML, tabsize=2, numbers=left, numberstyle=\tiny,
  stepnumber=2, numbersep=0pt, aboveskip=20pt, belowskip=20pt,
  xleftmargin=10pt}
\begin{lstlisting}[caption={Our \textsc{Bamboo} config file: simple.cfg},
  label=config]
<sandstorm> 
 <global>
   <initargs>
     node_id localhost:3200
   </initargs>
 </global>
 <stages>
   <Network>
     class bamboo.network.Network
     <initargs>
     </initargs>
   </Network>

   <Router>
     class bamboo.router.Router
     <initargs>
       gateway_count 1
       gateway_0  localhost:3200
     </initargs>
   </Router>

   <SimpleStage>
     class bamboo.SimpleStage
     <initargs>
       debug_level 1
       mode sender
     </initargs>
   </SimpleStage>
 </stages>
</sandstorm>
\end{lstlisting}

As the format of the config file is easy to understand I will just
explain the components of it and their meanings.

In line 2--6 you see the definition of global arguments. Every stage can
access these arguments. You see here how to define the node\_id for
this stage, which consist of \emph{hostname:portnumber}. 

At Line 7 the stage configuration starts; we need the Network and the
Router stages to send messages over the \textsc{Bamboo} P2P system. You have to
define at least one gateway to join the P2P network. In this example we
join ourself, assuming that we are the first node in the network. If
you are running more than one node you should add the node of the
network you want to join, here. Take a look at the \texttt{init()} function
of \texttt{bamboo.router.Router} in order to get an overview of the
available Router options and how to use them. 

As you can easily see, the definition of a new stage starts with a Tag
containing the name of the stage (the classname) followed by the option
class where the stage loader is told which class to load.
For our stage \emph{SimpleStage} we have two arguments called
\texttt{mode}, which will later switch between sending messages or not,
and \texttt{debug\_level} which is used by \emph{StandardStage} to
switch on debugging messages.

\section{Walk-through}
\label{sec:walkthrough}
%% ==============================

In this section we will walk through the code of our
\emph{SimpleStage} (see Appendix~\ref{sec:source}) and explain the
important things in the example. This very simple stage will
send (text-) messages to the node with the closest node-ID to zero. So
fasten your seat-belt and let's go.

\subsection{Class declaration and variables}
\label{sec:head}
%% ==============================

Let's skip the first 16 lines, you should be familiar with that. As I
mentioned before we extend from \texttt{bamboo.util.StandardStage} (it
might be a good idea to risk a glimpse into its source code sometime
later). \emph{StandardStage} implements
\texttt{seda.sandStorm.api.EventHandlerIF} and provides us a logging
interface and some functions you maybe find useful when writing your
own stage. When we use such a function from \emph{StandardStage} I
will point it out to you.

In lines 20 to 21 we help ourself to an application ID. It is used
like a TCP/UDP port number in \textsc{Bamboo}. When using the \textsc{Bamboo} routing
messages you have to specify the application ID you want to send to
(will almost every time be your own ID). To get a unique number for our
application (that is what a stage is, a \textsc{Bamboo} application), we use a
hash of our class name.


\subsection{Our own Payload}
\label{sec:payload}
%% ==============================

We are now talking about lines 28 to 46. As you can see, we define a
class named \emph{Payload} that implements
\texttt{ostore.util.QuickSerializable}. This is how to create a
payload for your messages you want to send. The payload has to be
serialized before sending and deserialization afterwards.

In line 39,\\
\lstset{language=Java, emph={serialize}, emphstyle=\textbf,
  emph={[2]Payload}, emphstyle={[2]\emph}}
\lstinline$  public void serialize(OutputBuffer b)$ \\
we have to add each variable we want to be transmitted to the output
buffer. The counterpart is the constructor,\\
\lstinline$  public Payload(InputBuffer buffer)$ \\
which deserializes the buffer back to the original variables. In our
example there is just the message itself (a string) that we need to
add to the buffer and save back from it.

Take a look into \texttt{ostore.util.InputBuffer} and
\texttt{ostore.util.OutputBuffer} to see which types and objects are
supported.


\subsection{A new Event type}
\label{sec:newevent}
%% ==============================

In line 49, we create a new message or event type (I think both notions
are OK). To create one, we need to implement the
\texttt{seda.sandStorm.api.QueueElementIF} interface which is in fact
empty, so we do not need to implement any functions. Normally you want
to add some content to the event like a payload (see
section~\ref{sec:payload}) and some other variables. In this case we
use this event to alarm us after a certain time; and do not need
any further functionality. There are plenty of examples in the \textsc{Bamboo}
source tree showing how to create your own event type.


\subsection{The constructor}
\label{sec:constructor}
%% ==============================

The important thing in our constructor is the setting up of the
\texttt{event\_types[]} array. In this array we save the classes of the
events we want to listen to (these events will be forwarded to our
\texttt{handleEvent()} function). \emph{StandardStage} defines two
more arrays for this purpose called \texttt{inb\_msg\_types[]} and
\texttt{outb\_msg\_types[]}. As we do not need to use them at this
point we will just cold-shoulder them for the moment. But I will talk
about them in Appendix~\ref{sec:networkmessage}.

So you can see the three events we registered to here. The last one we
already know: It is our own event \emph{Alarm} we create a few lines
above. The other two events are provided by \textsc{Bamboo} itself,
\texttt{seda.sandStorm.api.StagesInitializedSignal} is passed to us
when all the stages of our node finished their \texttt{init()}, so it
is a signal that we now can use the \textsc{Bamboo} system.\\
\texttt{bamboo.api.BambooRouteDeliver} is the event we get when
someone routed a message to our node through \textsc{Bamboo}.

So remember: You need to add your events you want to receive into these
arrays. We will see in section~\ref{sec:handleevent} what to to with these
events and how the whole thing works.

Another important thing is shown in line 61. In order to use them we
need to register our payloads to \textsc{Bamboo} by passing the class as an
argument:\\
\lstset{emph={register_type},emphstyle=\textbf}
\lstinline$  register_type(Payload.class);$


\subsection{Initialization}
\label{sec:init}
%% ==============================

When sandStorm starts all the stages it calls the \texttt{init()}
function of each stage and passes the data from the configuration file
to it.\\
The first thing we are doing is calling\\
\lstset{language=Java, emph={init,config_get_string,config_get_int,config_get_boolean}, emphstyle=\textbf}
\lstinline$  super.init(config);$ \\
which passes the configuration data to the \texttt{init()} function of
\emph{StandardStage}. If you take a glimpse into \emph{StandardStage}
you will see that it does two things:

\begin{itemize}
\item Setting the debug-level for the logger
\item Subscribing the events
\end{itemize}

The latter one is especially important for us -- \emph{StandardStage}
subscribes our events to \textsc{Bamboo}. I do not want to go more into detail:
If you want or need to know more about that refer to the
\emph{StandardStage} code and Appendix~\ref{sec:networkmessage}.

Now it is time to get the config options we added to the config file
in section~\ref{sec:config}. There are several functions to get the
config data, in line 79 we use\\
\lstinline$  config_get_string(config, "mode")$ \\
to retrieve a string from the option \emph{mode}. There are more
functions like that, e.g.\\
\lstinline$  config_get_int(config, "<name>")$ or\\
\lstinline$  config_get_boolean(config, "<name>")$,\\
\emph{StandardStage} will tell you more about these functions.

In the following lines we set the boolean \texttt{sender} to
\emph{true} if the config makes us to a sender, or \emph{false}
otherwise. 


\subsection{Event handling}
\label{sec:handleevent}
%% ==============================

\texttt{handleEvent(QueueElementIF elem)} is called every time an event
shows up at our node. You will only get events you have subscribed to
(see section~\ref{sec:constructor}) and -- if it is a \textsc{Bamboo} message
like \texttt{bamboo.api.BambooRouteDeliver} -- those that match your
application ID, so the purpose of this function is to handle the
event depending on which eventtype we received.

At this point a very important note: \textsc{Bamboo} is a singe-threaded
event-driven application (see the ``Programmers guide''
on~\cite{bamboo}). Because of this, you have to avoid 
blocking code in your event handler --- even more, you should hurry up
in your event handler because the whole (not only your stage, the
whole node) will be blocked until you finished your handling.

So let's go through this step by step:


\subsubsection{While not initialized}
\label{sec:handleevent:sec:preinit}
%% ==============================


So let's look at lines 90 to 111. As you will guess from the
\texttt{if(initialized)} in line 90 the here occurring events will
only appear during initialization.\\
One moment, we have done all initialization in the \texttt{init()}
function already. Well, that is right, but there are still some things
to do before our stage can to its regular work.

SandStorm will pass the
\emph{StagesInitializedSignal} to us as soon as
every stage has finished its \texttt{init()} function. So at this
point we can really use \textsc{Bamboo} for sending messages. And we have to, in
order to register our application ID (our ``port'') to \textsc{Bamboo} telling
it that we want to receive events for that specific ID. This is been
done by sending a \texttt{bamboo.api.BambooRouterAppRegReq}, passing
as the first argument our application ID, and as the last one, a
reference to our sink (that is our event queue).

By calling \texttt{dispatch(message)} the message is passed to the
so-called \emph{classifier}, which decides which stage should get this 
message (remember that the \emph{StandardStage} subscribed to our events
during \texttt{init()} from the classifier). We will use this function
every time we want to send (dispatch) a message.

If the application registration succeeds (which it should) a
\texttt{bamboo.api.BambooRouterAppReqResp} message will be passed to
our sink. Now we are fully registered to the system, and set
\texttt{initialized} to \emph{true}. At line 109 you will see, that we put all
occurring events during initialization into a LinkedList. Now after we
are really up, we process all the events from this list.

In line 106 you will notice a new command:\\
\lstset{language=Java, emph={dispatch_later}}
\lstinline$  dispatch_later(message, time);$\\
It is almost the same dispatch() command as before but you can now
specify a duration in milliseconds which tells the classifier how long
to wait till delivering the message. We use this command to set a
alarm timer to awake us after 10 seconds.


\subsubsection{BambooRouteDeliver}
\label{sec:handleevent:sec:deliver}
%% ==============================

When a message was sent to us routed by \textsc{Bamboo} (means using the
\texttt{bamboo.api.BambooRouteInit} message which we will get to know
in section~\ref{sec:handleevent:sec:alarm}), we will receive a
\texttt{BambooRouteDeliver} message.\\
In line 116 we cast the event to get access to its data. As you see we
assume that there is our \emph{Payload} attached as a payload and
print out the string saved in it. There are some more fields in this
message: Please take a look into the source for more information.


\subsubsection{Alarm}
\label{sec:handleevent:sec:alarm}
%% ==============================

When we are woken by our own \emph{Alarm} message we send a new text
message through \textsc{Bamboo} to the node with the lowest available node ID,
and start another alarm timer. So the lines 132 and 135 should be
familiar, but not line 126 and the following.

In these line you see how to let \textsc{Bamboo} route a message for you. To
tell \textsc{Bamboo} that, you have to use a \texttt{bamboo.api.BambooRouteInit}
message. The arguments are:

\begin{itemize}
\item destination node ID
\item application ID
\item use upcalls?
\item use iterative routing (or recursive)?
\item the payload to be transmitted
\end{itemize}

As you should already know \textsc{Bamboo} uses 160 bit node IDs as addresses
(like Pastry). To save such a long number in Java we have to
use a \texttt{BigInteger} object. In this example we route the message
to the nodeID 0 (zero).As you should know, when this node
does not exists \textsc{Bamboo} routes the message to the node with the closest
node ID.\\
The thing with the application ID is known since
section~\ref{sec:head}, so no more words about that.

If you pass \emph{true} as the next argument, \textsc{Bamboo} will make upcalls
to every interim node on the way of the message to its
destination. These nodes will receive a
\texttt{bamboo.api.BambooRouteUpcall} event (if they subscribed to
this event and if they have the same application ID of course) and can
change the message or react on it. And they need to continue the
routing by dispatching a \texttt{bamboo.api.BambooRouteContinue}. I
will not describe more of this mechanism: It is easy to understand and
familiar in every P2P system I know. Take a glimpse into the two
mentioned classes for more information.

The switch for choosing between iterative and recursive routing is
deprecated, so just say \emph{false} here.

Last but not least, we add the payload we want to transmit. That's it,
so easy to send a message over \textsc{Bamboo}.


\section{Finally}
\label{sec:final}
%% ==============================

Finally you want to run your stage. So compile \emph{SimpleStage} and
save the simple.cfg file to the appropriate place. Change (if not
already done) to the \textsc{Bamboo} directory and type:

\lstset{language=Perl, basicstyle=\small, numbers=none, aboveskip=10pt, belowskip=10pt}
\begin{lstlisting}
./bin/run-java bamboo.lss.DustDevil\ 
                            simple.cfg
\end{lstlisting}

This will start \emph{DustDevil} which uses \emph{SandStorm} to parse the
configuration file ``simple.cfg'' and starting the stages included in
the it. You should see the in the output how the messages are sent and
received.\\
You may start more nodes by adapting ``simple.cfg''.

So that's it: If you understand the things we used in this tutorial
you are ready to write your own one or start reading through the code
of the other stages shipped with \textsc{Bamboo}. You will find these structure
in almost every stage (beside the stages that uses the ASync event
model).

In Appendix~\ref{sec:networkmessage} you will find instructions how to
send messages direct to a node without routing them through
\textsc{Bamboo}. Appendix~\ref{sec:source} contains the full source code for
the \emph{SimpleStage}.

\begin{thebibliography}{99}
\bibitem{bamboo} \url{http://www.bamboo-dht.org}
\bibitem{pastry} \url{http://research.microsoft.com/~antr/Pastry/}
\bibitem{tapestry} \url{http://www.cs.berkeley.edu/~ravenben/tapestry/}
\bibitem{oceanstore}
  \url{http://oceanstore.org/publications/papers/abstracts/iptps03-api.html}
\end{thebibliography}

\appendix
\section*{Appendix}
\label{sec:appendix}
%% ==============================

\section{Sending messages direct to a node}
\label{sec:networkmessage}
%% ==============================

Sometimes you might not want to use \textsc{Bamboo} to route a message to a node,
but instead send it ``direct'' to a specified address (e.g. an IP address),
and not to its closest node (I set the word direct into quotation
marks because the message will still be routed if the destination is
not physical connected to our node). For this reason you want to use
\texttt{ostore.network.NetworkMessage}.

Let's have a quick look at \texttt{ostore.util.NodeId}. \emph{NetworkMessage}
uses this class for addresses. \emph{StandardStage} creates one for your node,
the variable name is \texttt{my\_node\_id}. Take a look into
\texttt{ostore.util.NodeId}: To create a new NodeId object you need just
the IP address and the port number. You will need to have such an object
for the destination node you want to send your message to.


\subsection{The Network stage}
\label{sec:networkmessage:sec:networkstage}
%% ==============================

So let's go through it step by step:

First, create a new \emph{NetworkMessage} object, e.g.:
\lstset{language=Java, aboveskip=10pt, belowskip=10pt, numbers=none,
  emph={NodeId,NetworkMessage}, emphstyle=\emph}
\begin{lstlisting}  
NodeId dest = 
 new NodeId(3200, InetAddress.
        getByName("localhost"));
NetworkMessage nm = 
 new NetworkMessage(dest, false);
\end{lstlisting}
If you add your sink
(\emph{StandardStage} saves it to \texttt{my\_sink}) to
\texttt{comp\_q}, the Network stage will send you a
\texttt{ostore.network.NetworkMessageResult} which will tell you if
transmission was successful or not. If you put something in
\texttt{user\_data} this will also be passed to you together with the
NetworkMessageResult event.

Dispatch the event, and that's it -- you've sent the message
directly. If you want to send any payload you will have to create
a new event type extending from NetworkMessage and adding a variable for
the payload (of type \texttt{ostore.util.QuickSerializable}).


\subsection{Again: Event types}
\label{sec:networkmessage:sec:eventtypes}
%% ==============================

So how does this work? For explanation we have to step back to
section~\ref{sec:constructor} just for a moment where I mentioned that
there are two more arrays for events called \texttt{outb\_event\_types[]}
and \texttt{inb\_event\_types[]}.  

When \emph{StandardStage} subscribes to these messages it also adds a filter to
each event listed in these arrays. It checks that there is a boolean
called \texttt{inbound} in each event and it tells the classifier that this stage
only wants to receive these events if \texttt{inbound} is \emph{true} (for
\texttt{inb\_event\_types[]}) or \emph{false} in the other case (for
\texttt{outb\_event\_types[]}).\\
This is one possible filter (like the one you have with \textsc{Bamboo} messages
that are filtered by your application ID), take a look into
\texttt{bamboo.util.StandardStage} for an example how to use filters.

The thing with the Network stage is that it listens to \emph{NetworkMessages}
with \texttt{inbound} set to \emph{false} and sends them to the destination
represented by the NodeId object.


\subsection{Receiving a NetworkMessage}
\label{sec:networkmessage:sec:receiving}
%% ==============================

So we sent a \emph{NetworkMessage} but we have not done anything to receive
them (remember we have to subscribe to each message we want to listen
to).

The Network stage sends the message to the Network stage of the
destination node. This stage then sets \texttt{inbound} to \emph{true} and sets
the new destination to its own node ID and dispatches it. Local stages
that subscribe to this type of message will now receive it:
\lstset{language=Java, aboveskip=10pt, belowskip=10pt, numbers=none}
\begin{lstlisting}
inb_event_types[] = 
    new Class[]{NetworkMessage.class}
\end{lstlisting}

The original sender can be found in the variable named \texttt{sender}
(not in \texttt{source}!).


\subsection{NetworkMessageResult}
\label{sec:networkmessageresult}
%% ==============================

I mentioned this before: If you provide your sink to the \emph{NetworkMessage} it
will send you back an \texttt{ostore.network.NetworkMessageResult} event
which will inform you about successful or unsucessful
transmission. Therefore check the variable \texttt{success}, the
\texttt{user\_data} is also passed back to you in this message so that you
can identify the message that failed. Of course you need to subscribe
to this event type, so add it to \texttt{event\_types[]}.


\onecolumn
\section{The sourcecode of SimpleStage}
\label{sec:source}
%% ==============================

\lstset{language=Java, basicstyle=\small, showstringspaces=false,
  numbers=left, numberstyle=\tiny, numberfirstline=true, stepnumber=5,
  numbersep=5pt}

\lstinputlisting{SimpleStage.java}




\end{document}

