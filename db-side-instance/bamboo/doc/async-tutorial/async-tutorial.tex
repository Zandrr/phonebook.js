\documentclass[11pt]{article}

\usepackage{listings}
\usepackage{url}
\usepackage{verbatim}
\usepackage{times}
\usepackage{courier}

\setlength{\textwidth}{6.5in}
\setlength{\textheight}{9.0in}
\setlength{\topmargin}{0.0in}
\setlength{\oddsidemargin}{0.0in}

\title{A Programmer's Tutorial on Event-Driven Programming,\\
       Asynchronous Input/Output, and the Bamboo DHT}
\author{Sean C. Rhea\\ srhea@cs.berkeley.edu}

\begin{document}

\maketitle

\section{Preliminaries}

This document is a tutorial on three related topics: event-driven
programming, asynchronous input/output (I/O), and the Bamboo DHT.  I
assume you're already familiar with Java, including the new language
features for generics (called templates in C++) introduced in Java 1.5.
If you're not yet aware of what generics are or how they work, please
first read ``Using and Programming Generics in J2SE 5.0'' on Sun's web
page.\footnote{\url{http://java.sun.com/developer/technicalArticles/J2SE/generics/}}
Most of the example code shown in this tutorial is available on the
Bamboo website.\footnote{\url{WHAT}}  I encourage you to download, run,
modify, and re-run the example programs; there's no better way to learn
a new programming style or library.

\section{Introduction}

We'll start the tutorial with an example: assume that you are writing a
high-performance web server.  A main loop for the web server might look
something like this:

\begin{quote}
\lstset{language=Java, basicstyle=\small}
\begin{lstlisting}  
while (true) {
    Socket sock = acceptConnection();
    HttpRequest req = parseRequest(sock);
    HttpResponse resp = createResponse(req);
    sendResponse(sock, resp);
    sock.close();
}
\end{lstlisting}
\end{quote}

Presumably, this web server will have a high-bandwidth connection to the
Internet.  On the other hand, many of the clients of the web server may
have slow connections, meaning that your code will spend a lot of time
in the \texttt{sendResponse} function.  In effect, the performance of
your web server is limited by the bandwidth of whatever client it's
serving at any given time.  (That's bad.)

To get around this performance limitation, we need to introduce
\emph{concurrency}: the ability to serve more than one client at a time.
Historically, there have been at least two popular approaches to adding
concurrency to a program.  One approach is called \emph{multithreading}.
In this approach, the above code would be rewritten like this:

\begin{quote}
\lstset{language=Java, basicstyle=\small}
\begin{lstlisting}  
while (true) {
    final Socket sock = acceptConnection();
    Thread t = new Thread() {
        public void run() {
            HttpRequest req = parseRequest(sock);
            HttpResponse resp = createResponse(req);
            sendResponse(sock, resp);
            sock.close();
        }
    };
    t.start();
}
\end{lstlisting}
\end{quote}

This code is much better.  Since we have a separate thread for each
request, and since each thread will run independently of all the others,
it won't matter if one client has a very slow connection anymore; the
\texttt{sendResponse} function will still take a long time to complete
for that client, but it will run in parallel with the
\texttt{sendResponse} functions for all the other clients we're
servicing at the same time, rather than blocking them as before.

The above code is still not perfect, however.  First of all, creating a
new thread is an expensive operation in Java, and programs written like
the one above can often spend much of their time creating new threads.
While this problem can be solved with a construct called a \emph{thread
pool}, there is yet another problem with this style of code: most modern
operating systems do not perform well when the number of Java threads
running on them passes beyond several hundreds.  Why this performance
problem occurs is not important here, but it does mean that we can't
expect a web server written in the above style to serve more than a
couple of hundred clients at a time.  

To work around this performance problem, we need to change our style of
programming to what is often called \emph{event-driven}
style.\footnote{If you've ever programmed using Java's Swing graphics
library, you've already done some event-driven programming.}  The main
idea of this style of programming is that given any function call that
might block on input or output, such as the \texttt{sendResponse}
function above, we split that function into a \emph{request event} and a
\emph{reponse event}, normally represented as function objects.  For
example, we might define new versions of \texttt{parseRequest} and
\texttt{sendResponse} like this:

\begin{quote}
\lstset{language=Java, basicstyle=\small}
\begin{lstlisting}  
public interface ParseDone {
    public void run(HttpRequest req);
}
void parseRequest(Socket sock, ParseDone done) { ... }
void sendResponse(Socket sock, HttpResponse resp, Runnable done) { ... }
\end{lstlisting}
\end{quote}

The idea is that \texttt{parseRequest} will now return immediately after
being called, and sometime later, when the request has actually been
read off the network and parsed, the \texttt{run} function of the
\texttt{ParseDone} object passed into \texttt{parseRequest} will be
called.  Likewise, \texttt{sendResponse} will also return immediately,
and the \texttt{run} function of the \texttt{Runnable} object passed
into \texttt{sendResponse} will be called later, once the response has
been sent.  

Ignore for a moment how these new versions of \texttt{parseRequest} and
\texttt{sendResponse} are implemented, and consider only how we might
use them.  For example, we could rewrite the web server's main loop like
this:

\begin{quote}
\lstset{language=Java, basicstyle=\small, numbers=left, numberstyle=\tiny, 
        numberfirstline=true, stepnumber=1, numbersep=5pt}
\begin{lstlisting}  
while (true) {
    final Socket sock = acceptConnection();
    ParseDone parseDone = new ParseDone() {
        public void run(HttpRequest req) {
            HttpResponse resp = createResponse(req);
            Runnable sendResponseDone = new Runnable() {
                public void run() { sock.close(); }
            };
            sendResponse(sock, resp, sendResponseDone);
        }
    };
    parseRequest(sock, parseDone);
}
\end{lstlisting}
\end{quote}

Wow, that's some ugly code!  But look past its ugliness for the moment,
and let's think about what it does.  First, one line 2 we accept a
connection as before.  Then, we decide what we're going to do once we've
parsed a request and write it as a \texttt{ParseDone} object on lines
3--11.  Next, we call \texttt{parseRequest} on line 12.  That will parse
the request, and then call \texttt{run} on line 4.  In turn,
\texttt{run} creates a response (line 5) and decides what it's going to
do once the response is sent, encoding that knowledge as an object of
type \texttt{Runnable}.  It then calls \texttt{sendResponse}, which
sends the response and then calls \texttt{run} on line 7, which closes
the socket.  Whew!

You're probably thinking that the above code is one seriously convoluted
way to write a web server.  In part you're right, although we'll
improve on the readability of the above code later in this tutorial.  At
the same time, however, notice that we've both eliminated the problem of
clients blocking on one another and threads from the program.  Since
there are no longer any blocking calls, each run of the while loop will
now happen as quickly as possible, bringing us right back to the call to
\texttt{acceptConnection} where we can begin helping the next client
right away.  And if you trust me for the moment that we can implement
the new versions of \texttt{parseRequest} and \texttt{sendResponse} with
only a single thread, then we've also eliminated our previous need for
multiple threads.

The remainder of this document is in several parts.  First, I'll
describe the event model that Bamboo uses, and the various Bamboo
classes that make event-driven programming easier.  I won't describe how
these classes are implemented in this tutorial, for that you'll have to
read the source if you're interested.  (It's not necessary to understand
the source in order to use it.)  Next, I'll describe how to use the
Bamboo messaging layer and DHT router, so that you can build your own
DHT applications on top of Bamboo.  

\section{The Bamboo Event Model}

In this section I'll get into the full details of the event model Bamboo
uses, staring from the beginning.

\subsection{Callbacks and \texttt{registerTimer}}

Above we saw a use of the interface \texttt{java.lang.Runnable}, which
is defined like this:
\begin{quote}
\lstset{language=Java, basicstyle=\small}
\begin{lstlisting}  
public interface Runnable { public void run(); }
\end{lstlisting}
\end{quote}
This kind of interface is often called a \emph{callback}; you pass an
object of type \texttt{Runnable} to some other function so that it can
``call you back'' later by calling the \texttt{run} function.  Above, we
passed one such object as an argument to the \texttt{sendResponse}
function.  

Events in Bamboo are implemented as either functions or callbacks.  This
idea is perhaps best explained by example.  Bamboo includes a class,
\texttt{bamboo.lss.ASyncCore} that contains two functions,
\texttt{registerTimer} and \texttt{asyncMain}.  You use them like this:

\begin{quote}
\lstset{language=Java, basicstyle=\small, numbers=left, numberstyle=\tiny, 
        numberfirstline=true, stepnumber=1, numbersep=5pt}
\lstinputlisting{DelayedHelloWorld.java}
\end{quote}

When you run this program, it will pause for five seconds and then print
``Hello, World!'' before exiting.  Let's break down the code a line at a
time.  

First, lines 2--7 define a callback, \texttt{printCallback}, that prints
``Hello, world.'' to standard output when it's called.  Line 9 creates
an instance of \texttt{ASyncCore}, and line 10 tells that object to call
\texttt{printCallback} in 5 seconds (5,000 milliseconds).  Finally, line
11 calls \texttt{asyncMain}; once this function is called,
\texttt{ASyncCore} is in charge.  The function \texttt{asyncMain} never
exits, it simply loops forever, calling the callbacks registered with
\texttt{registerTimer} when required.  (That's why we call \texttt{exit}
on line 5; otherwise, the program would run forever.  Remove that call
and try it to see what I mean.)

In your event-driven programs, you will usually create an object of type
\texttt{ASyncCore}, perform some other setup including registering timer
events, and then call \texttt{asyncMain}.  

\subsection{Passing Arguments to Callbacks}

Let's say that instead of wanting to print ``Hello, World!'' after five
seconds, you instead wanted to print whatever string was passed in on the
command line.  You might try something like this:

\begin{quote}
\lstset{language=Java, basicstyle=\small, numbers=left, numberstyle=\tiny, 
        numberfirstline=true, stepnumber=1, numbersep=5pt}
\lstinputlisting{BadDelayedEcho.java}
\end{quote}

Here we've used what amounts to a global variable, \texttt{str}, to pass the
string to echo from the \texttt{main} function to \texttt{printCallback}.  Bad
programmer!  Maybe you're crafty, though, and you've already thought of a way
around using a global variable, such as this:

\begin{quote}
\lstset{language=Java, basicstyle=\small, numbers=left, numberstyle=\tiny, 
        numberfirstline=true, stepnumber=1, numbersep=5pt}
\lstinputlisting{BetterDelayedEcho.java}
\end{quote}

In this program, we've removed the global variable by defining
\texttt{printCallback} inside of \texttt{main}, and thereby putting
\texttt{str} within its scope.  Pretty crafty, but it has one
big downside: what if \texttt{printCallback} itself defined another
callback, which in turn defined another callback, and so on?
Pretty soon you'd suffer death through excessive indentation.  

\subsection{The \texttt{curry} Function}

To pass arguments to callbacks without using so much indentation, we'll
introduce a technique from \emph{functional} programming languages (such
as Lisp and ML) called \emph{currying}.  In such languages, given a
function that takes two arguments and a value for one argument, you can
produce a new function that takes only the second argument.  For
example, using Java-like syntax, we might have:

\begin{quote}
\lstset{language=Java, basicstyle=\small}
\begin{lstlisting}  
public int add(int left, int right) { return left + right; }
public int add42(int right) = curry(add, 42);
...
int j = add42(7);
\end{lstlisting}
\end{quote}

In other words, we curry the function \texttt{add} with the value 42 to
produce a new function, \texttt{add42}, that takes one argument and
returns the value of that argument plus 42.  Right now this technique
might seem a little pointless, but bear with me; it will be a useful
technique for passing arguments to callbacks.

It turns out that we can use Java's support for generics (called
templates in C++) to implement currying.  All we need is a way to refer
to functions as objects, and we'll use callbacks for this purpose.
Because of this decision, we will sometimes refer to callbacks as
\emph{function objects}.

We already have a callback that takes no arguments: the
\texttt{Runnable} interface used above.  What we'll do is introduce
another type of callback that takes one argument like this:

\begin{quote}
\lstset{language=Java, basicstyle=\small}
\begin{lstlisting}  
public interface Thunk1<T> { void run(T t); }
\end{lstlisting}
\end{quote}

\emph{Thunk} is the term used in functional programming for a function
that has return type void.  We've called the above interface
\texttt{Thunk1} because it takes one argument and returns void.  We can
create a new \texttt{Thunk1} object that takes an integer argument like
this:

\begin{quote}
\lstset{language=Java, basicstyle=\small}
\begin{lstlisting}  
Thunk1<Integer> intThunk = new Thunk1<Integer>() { 
    void run(Integer i) { System.out.println(i); }
}
\end{lstlisting}
\end{quote}
Moreover, we can create a \texttt{curry} function that will take a
\texttt{Thunk1<T>} and an object of type \texttt{T} and return a
\texttt{Runnable} object like this: 

\begin{quote}
\lstset{language=Java, basicstyle=\small}
\begin{lstlisting}  
public static <T> Runnable curry(final Thunk1<T> f, final T t) {
    return new Runnable() { public void run() { f.run(t); } };
}
\end{lstlisting}
\end{quote}
We can then use this function to create a function that prints 42 like
this:

\begin{quote}
\lstset{language=Java, basicstyle=\small}
\begin{lstlisting}  
Runnable print42 = curry(intThunk, new Integer(42));
print42.run(); // prints "42" to standard output
\end{lstlisting}
\end{quote}

Bamboo contains a class, \texttt{bamboo.util.Curry}, that has
definitions for \texttt{Thunk1} through \texttt{Thunk9}, and the
appropriate \texttt{curry} functions for all combinations of them.  You
can, for example, call \texttt{curry} with a \texttt{Thunk7} and two
arguments to get a \texttt{Thunk5}.  To use all of these thunks and
curry functions in your code, include this line at the top of your
source file:

\begin{quote}
\lstset{language=Java, basicstyle=\small}
\begin{lstlisting}  
import static bamboo.util.Curry.*;
\end{lstlisting}
\end{quote}

Now, if you remember, we were trying to improve the
\texttt{BetterDelayedEcho} program above.  Here's a new version using
thunks and curries:

\begin{quote}
\lstset{language=Java, basicstyle=\small, numbers=left, numberstyle=\tiny, 
        numberfirstline=true, stepnumber=1, numbersep=5pt}
\lstinputlisting{DelayedEcho.java}
\end{quote}

Lines 3-8 of this program create a new callback, this time one that takes a
string argument and prints it to standard output.  
Unfortunately, the \texttt{registerTimer} function only takes arguments of
type \texttt{Runnable}, so we can't pass it our new version of
\texttt{printCallback} directly.  We can, however, call \texttt{curry}
with \texttt{printCallback} and the string we want passed to it as
arguments to get an object of type \texttt{Runnable}, and we can then
pass that to \texttt{registerTimer}.
In other words, line 15 says, ``after
five seconds, call \texttt{printCallback} with the value in
\texttt{args[0]}.''  Pretty cool, huh?

\subsection{Using Channels}

Timers are only one of the two main classes of callbacks that
\texttt{ASyncCore} manages.  The other class is for sockets, Unix's
stream-based abstraction of the network.  In Java 1.4 and later, sockets
are accessed through objects called \emph{channels}.  There are
\texttt{SocketChannels} for TCP clients, \texttt{ServerSocketChannels}
for TCP servers, and \texttt{DatagramChannels} for UDP sockets.  

The class \texttt{java.nio.channels.SelectionKey} defines the conditions you
can wait on with a channel: \texttt{OP\_ACCEPT}, \texttt{OP\_CONNECT},
\texttt{OP\_READ}, and \texttt{OP\_WRITE}.  To wait on any one of these
conditions, call \texttt{registerSelectable} in \texttt{ASyncCore}; to
stop waiting on them, call \texttt{unregisterSelectable}.  Here's an
example:

\begin{quote}
\lstset{language=Java, basicstyle=\small, numbers=left, numberstyle=\tiny, 
        numberfirstline=true, stepnumber=1, numbersep=5pt}
\lstinputlisting{HttpGet.java}
\end{quote}

This program takes the name and port number of an HTTP server on the command
line, connects to it using TCP, and sends the string ``GET / HTTP/1.0''
followed by two carriage-return, line-feed pairs.  This string will cause any
compliant HTTP server to return the resource named by ``/'', usually the file
\texttt{index.html}.

Important things to note include: on line 84 we call
\texttt{configureBlocking(false)} on the \texttt{SocketChannel} object.  If
this is not done, calls to read or write the socket may block.  
Control will next move to the \texttt{acceptCallback} function once the TCP
connection is established.  After
connecting on line 17, we unregister the accept callback on line 18.  We then
create a \texttt{ByteBuffer} object to hold the string we want to write, and
register the the write callback function.

The function \texttt{SocketChannel.write} returns zero if no data can be
written at the moment, and throws an \texttt{IOException} if the socket is
closed.  Once we have written all the data in the buffer, we unregister the
write callback and register the read callback (lines 40--44).

Unlike \texttt{write}, \texttt{SocketChannel.read} has two completion cases.
Like \texttt{write}, if the connection times out or some other error occurs,
\texttt{read} will throw an \texttt{IOException}.  Unlike \texttt{write},
if the remote node closes the connection normally, \texttt{read} will return
-1.  We read until either of these conditions occurs, echoing all bytes read
to standard output.

\begin{quote}
\lstset{language=Java, basicstyle=\small, numbers=left, numberstyle=\tiny, 
        numberfirstline=true, stepnumber=1, numbersep=5pt}
\lstinputlisting{EchoServer.java}
\end{quote}

%Bamboo's event-driven model is based on \texttt{libasync}, which is
%part of the Self-Certifying File
%System.\footnote{\url{http://www.fs.net/sfswww/}} This document, in
%turn, is based on the ``\texttt{libasync}
%tutorial''.\footnote{\url{http://www.pdos.lcs.mit.edu/6.824-2004/async/index.html}}
%If you're interested in the kind of functionality this document
%describes, but want to program in C++ instead, be sure to check them
%out.

\end{document}

